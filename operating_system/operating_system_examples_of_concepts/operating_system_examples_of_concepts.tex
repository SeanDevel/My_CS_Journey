\documentclass[a4paper]{book}

% Package for setting up A4 size and margins like Word's default
\usepackage[a4paper, margin=2.54cm]{geometry}

% Package for Chinese characters
\usepackage{ctex}

% Package for mathematical formulas
\usepackage{amsmath}
\usepackage{graphicx}

\setcounter{secnumdepth}{1}

% Package for code highlighting
% pip install Pygments
% pip install latexminted
\usepackage{minted}

\usepackage{silence}
\WarningFilter{latex}{UTF8}

\usepackage{xcolor}
\usepackage{mdframed}

\definecolor{green}{RGB}{230, 252, 238}
\newmdenv[
backgroundcolor=green,
linecolor=green
]{greenbox}

\begin{document}

\chapter{Process Management}

\section{Basic Concepts}

\subsection{Control and Describe Process}

\begin{greenbox}
The reason why the operating system introduces the concept of process is to enable programs to be executed concurrently in a multi-programming environment, and to control and describe the concurrently executed programs. Please give some examples to explain this in detail.
\end{greenbox}

操作系统之所以要引入进程的概念,是为了使程序在多道程序环境下能并发执行,并对并发执行的程序加以控制和描述。请举例具体说明。 

\subsubsection{Explanation of Processes in Operating Systems}

The concept of a \textbf{process} is introduced to allow programs to run concurrently in a multi-programming environment. A \textbf{process} is essentially a program in execution, which includes not just the program code but also the program's state, data, and resources. The operating system uses processes to manage and control the execution of multiple programs while ensuring resource sharing and isolation.

\rule{0.75\textwidth}{0.5pt}

\subsubsection{Examples of Processes and Their Role in Concurrent Execution}

\textbf{1. Web Server Handling Multiple Client Requests}

   - \textbf{Scenario}: A web server, like Apache or Nginx, handles multiple client requests simultaneously.
   
   - \textbf{How Processes Help}:
     
     - Each client request is handled as a separate process or thread. 
     
     - The operating system schedules these processes to run concurrently on available CPU cores, allowing the server to handle many clients simultaneously.
   
   - \textbf{Advantages}:
     
     - Resource sharing: Multiple processes can access shared resources like memory or files.
     
     - Isolation: If one client process crashes, it does not affect others.

\rule{0.75\textwidth}{0.5pt}

\textbf{2. Text Editor and Music Player Running Together}

   - \textbf{Scenario}: A user is writing a document in a text editor while listening to music using a media player.
  
   - \textbf{How Processes Help}:

     - The text editor and the music player are separate processes.

     - The operating system ensures that both processes get CPU time, switching between them rapidly (time-sharing).

   - \textbf{Advantages}:

     - The user perceives that both tasks are happening simultaneously.

     - Processes are independent: The music continues even if the text editor crashes.

\rule{0.75\textwidth}{0.5pt}

\textbf{3. Background File Download While Browsing the Internet}

   - \textbf{Scenario}: A browser downloads a file in the background while the user continues to browse.
   
   - \textbf{How Processes Help}:
     
     - The browser spawns a separate process for the file download.
     
     - This allows the main browser process to remain responsive for user interactions while the download process handles network operations.
   
   - \textbf{Advantages}:
     
     - Improved user experience: The browser remains responsive.
     
     - Modularity: The download process can be paused, resumed, or even terminated without affecting the main browser.

\rule{0.75\textwidth}{0.5pt}

\textbf{4. Compilation of Source Code}

   - \textbf{Scenario}: A developer is compiling a large program using `gcc` while editing another program.
   
   - \textbf{How Processes Help}:
     
     - The compilation process runs independently of the text editor.
     
     - The operating system schedules the compilation and the text editing processes to share CPU and memory.
   
   - \textbf{Advantages}:
     
     - Concurrency: Both tasks can progress simultaneously, improving productivity.
     
     - Isolation: Errors in the compilation process (e.g., syntax errors) do not affect the text editor.

\rule{0.75\textwidth}{0.5pt}

\textbf{5. Print Queue Management}

   - \textbf{Scenario}: Multiple users send documents to a shared printer in a networked office.
   
   - \textbf{How Processes Help}:
     
     - Each print job is treated as a separate process managed by the print spooler.
     
     - The operating system queues the print jobs and schedules them one by one, ensuring fair access.
   
   - \textbf{Advantages}:
     
     - Resource sharing: The printer is shared efficiently among all users.
     
     - Control: Users can pause or cancel their print jobs without interfering with others.

\rule{0.75\textwidth}{0.5pt}

\textbf{Key Features Enabled by Processes}


1. \textbf{Concurrency}: Processes allow multiple tasks to be executed seemingly at the same time.

2. \textbf{Resource Sharing}: Processes enable efficient sharing of system resources like CPU, memory, and I/O devices.

3. \textbf{Isolation}: Each process runs in its own address space, ensuring that errors or crashes in one process do not affect others.

4. \textbf{Synchronization and Communication}: The operating system provides mechanisms (e.g., semaphores, message passing) for processes to coordinate and share data when needed.

---

\textbf{Summary}

The introduction of processes allows operating systems to manage the complexities of concurrent execution. By using processes, the system can balance resource utilization, isolate tasks for stability, and enable efficient and responsive multi-tasking. Examples like web servers, text editors, media players, and print queues illustrate how processes are central to modern computing environments.

\subsubsection{Why Describing?}

\begin{greenbox}
What is the purpose of OS describing concurrent programs?
\end{greenbox}

The purpose of the operating system (OS) describing concurrent programs lies in its role as a \textbf{manager and mediator} for programs running simultaneously. Describing concurrent programs means that the OS needs to maintain and provide a \textbf{structured representation} of each program and its state. This is critical for ensuring proper execution, coordination, and resource management in a multi-programming environment. Below are the key purposes of this descriptive functionality:

\rule{0.75\textwidth}{0.5pt}

\textbf{1. Tracking the State of Each Program}

   - \textbf{Purpose}: The OS needs to know what each program is doing at any given time (e.g., running, waiting, or terminated).
   
   - \textbf{How It Works}:
   
     - The OS maintains a \textbf{process control block (PCB)} for each process, which describes its:
       
       - Unique identifier (PID)
       
       - State (e.g., running, ready, waiting)
       
       - Resource usage (e.g., memory allocated, open files)
       
       - CPU registers and program counter (to resume execution later)
   
   - \textbf{Why It's Important}:
     
     - Ensures that processes are resumed from the \textbf{correct point} after being paused or interrupted.
     
     In real-life operating systems, the \textbf{"correct point"} refers to the exact state of a process at the moment it is interrupted or paused. This includes:

1. \textbf{Program Counter (PC)}: The memory address of the next instruction to be executed.

2. \textbf{CPU Registers}: Values stored in the processor's registers at the time of interruption.

3. \textbf{Memory State}: The current data in the process's allocated memory.

4. \textbf{I/O State}: The status of any ongoing input/output operations.

When the OS resumes the process, it restores these states so that the process continues execution as if it had never been paused. This ensures continuity and correctness.

     - Allows the OS to manage transitions between states efficiently.

\textbf{1. Tracking the State of Each Program}

\textbf{2. Ensuring Fair Resource Allocation}

   - \textbf{Purpose}: Concurrent programs compete for resources like CPU, memory, and I/O devices. The OS must describe resource usage to allocate fairly and avoid conflicts.
   
   - \textbf{How It Works}:
     
     - The OS keeps tables describing:
       
       - Memory allocation for each process.
       
       - Open files and devices being used.
       
       - CPU scheduling priorities and timeslices.
   
   - \textbf{Why It's Important}:
     
     - Prevents resource starvation or deadlock.
     
     - Optimizes resource utilization across all running programs.

\rule{0.75\textwidth}{0.5pt}

\textbf{3. Facilitating Inter-Process Communication (IPC)}

   - \textbf{Purpose}: Many concurrent programs need to communicate or share data (e.g., a web browser process communicating with a file download process).
   
   - \textbf{How It Works}:
     
     - The OS provides descriptions of processes and their shared resources to facilitate:
       
       - Message passing
       
       - Shared memory regions
       
       - Synchronization primitives (e.g., semaphores, mutexes)
   
   - \textbf{Why It's Important}:
     
     - Enables cooperation and coordination among processes.
     
     - Prevents conflicts or inconsistencies when accessing shared data.
     
\rule{0.75\textwidth}{0.5pt}

\textbf{4. Supporting Debugging and Monitoring}

   - \textbf{Purpose}: Describing concurrent programs allows developers and administrators to debug and monitor their behavior.
   
   - \textbf{How It Works}:
     
     - The OS provides tools and data, such as:
       
       - Logs of process activity.
       
       - Resource usage statistics (e.g., CPU, memory, I/O).
       
       - Event tracing for identifying bottlenecks or errors.
   
   - \textbf{Why It's Important}:
     
     - Helps identify and resolve issues in concurrent systems.
     
     - Provides insights into system performance and optimization opportunities.

\rule{0.75\textwidth}{0.5pt}

\textbf{5. Handling Synchronization and Coordination}

   - \textbf{Purpose}: The OS uses descriptions to manage dependencies and coordination between concurrent programs.
   
   - \textbf{How It Works}:
     
     - The OS describes:
       
       - Process dependencies (e.g., one process waiting for another).
       
       - Conditions for transitions (e.g., process A starts only after process B completes).
  
   - \textbf{Why It's Important}:
     
     - Avoids deadlocks and race conditions.
     
     - Ensures that interdependent programs operate in a consistent and predictable manner.

\rule{0.75\textwidth}{0.5pt}

\textbf{6. Implementing Security and Isolation}

   - \textbf{Purpose}: The OS uses descriptions to enforce security and prevent one program from interfering with another.
   
   - \textbf{How It Works}:
   
     - The OS maintains descriptions of:
       
       - Each process's permissions (e.g., access to files or devices).
       
       - Memory regions allocated to each process.
   
   - \textbf{Why It's Important}:
     
     - Protects processes from unauthorized access or modification.
     
     - Prevents security vulnerabilities arising from concurrent execution.
     
\rule{0.75\textwidth}{0.5pt}

\textbf{Example: Describing Processes in a Multi-Program Environment}

Imagine a system where a user is running:

1. A text editor.

2. A music player.

3. A background file download.

The OS describes each program as follows:

- \textbf{Process Control Block (PCB)}: Tracks their state (e.g., text editor is running, music player is ready, file download is waiting for I/O).

- \textbf{Resource Allocation}: Describes CPU time, memory, and I/O devices allocated to each program.

- \textbf{Dependencies and Communication}: Tracks whether the file download depends on network availability or if the text editor opens a file that another program is writing to.
By maintaining and using these descriptions, the OS ensures smooth execution, resource sharing, and responsiveness.

\rule{0.75\textwidth}{0.5pt}

\textbf{Summary}

The purpose of the OS describing concurrent programs is to:

- Track their state and behavior.

- Allocate resources efficiently.

- Facilitate inter-process communication.

- Support debugging and monitoring.

- Coordinate dependencies and synchronization.

- Enforce security and isolation.

These descriptions enable the OS to manage complex interactions and ensure that the system remains stable, efficient, and user-friendly, even when multiple programs execute concurrently.


\subsection{Discontinuity}

\begin{greenbox}
Concurrent execution of programs will produce discontinuity characteristics. When programs are executed concurrently, they share system resources and therefore need to cooperate with each other to complete the same task. This creates a mutual constraint relationship between these concurrently executed processes, which causes discontinuity to occur during the execution of the process. Please give an example to explain this in detail.
\end{greenbox}

程序并发执行会产生间断性特征。程序在并发执行时,由于他们共享系统资源,因此为完成同一任务需要相互合作。这使得这些并发执行的进程之间形成了相互制约关系,从而使得进程在执行期间出新出现间断性。请举例具体说明。 

\subsubsection{Producer-Consumer Problem with a Shared Buffer}

\textbf{Scenario}

In the \textbf{Producer-Consumer Problem}, a producer generates data items and places them in a shared buffer, while a consumer retrieves and processes those items. They operate concurrently and share system resources such as memory (the buffer). 

\textbf{Shared Resources}

- A \textbf{buffer} with a fixed size \( N \), which can hold up to \( N \) data items.

- Synchronization primitives (e.g., semaphores, mutexes) to avoid conflicts.

\textbf{Discontinuity Characteristics}
The producer and consumer do not execute continuously because they are mutually constrained by the state of the shared buffer:

1. \textbf{Producer Constraints (Buffer Full):}

   - If the buffer is full, the producer must wait before adding more data. This introduces a discontinuity in the producer's execution because it cannot proceed until space is available in the buffer.

2. \textbf{Consumer Constraints (Buffer Empty):}

   - If the buffer is empty, the consumer must wait for the producer to add data. This introduces a discontinuity in the consumer's execution because it cannot proceed without new data.

\textbf{Code Illustration (Simplified Pseudocode)}

\begin{minted}[frame=lines, fontsize=\normalsize]{c}
#include <pthread.h>
#include <semaphore.h>
#include <stdio.h>

#define BUFFER_SIZE 5
int buffer[BUFFER_SIZE];
int in = 0, out = 0;

// Semaphores for synchronization
sem_t empty; // Counts empty slots
sem_t full;  // Counts filled slots
pthread_mutex_t mutex; // Mutual exclusion

void *producer(void *arg) {
    for (int i = 0; i < 10; i++) {
        sem_wait(&empty); // Wait if buffer is full
        pthread_mutex_lock(&mutex);

        // Critical section: Add an item to the buffer
        buffer[in] = i;
        printf("Produced: %d\n", i);
        in = (in + 1) % BUFFER_SIZE;

        pthread_mutex_unlock(&mutex);
        sem_post(&full); // Signal consumer that buffer has data
    }
    return NULL;
}

void *consumer(void *arg) {
    for (int i = 0; i < 10; i++) {
        sem_wait(&full); // Wait if buffer is empty
        pthread_mutex_lock(&mutex);

        // Critical section: Remove an item from the buffer
        int item = buffer[out];
        printf("Consumed: %d\n", item);
        out = (out + 1) % BUFFER_SIZE;

        pthread_mutex_unlock(&mutex);
        sem_post(&empty); // Signal producer that buffer has space
    }
    return NULL;
}

int main() {
    pthread_t prod, cons;
    sem_init(&empty, 0, BUFFER_SIZE);
    sem_init(&full, 0, 0);
    pthread_mutex_init(&mutex, NULL);

    pthread_create(&prod, NULL, producer, NULL);
    pthread_create(&cons, NULL, consumer, NULL);

    pthread_join(prod, NULL);
    pthread_join(cons, NULL);

    sem_destroy(&empty);
    sem_destroy(&full);
    pthread_mutex_destroy(&mutex);

    return 0;
}
\end{minted}

\textbf{Explanation of Discontinuity}

1. \textbf{Buffer Full:}

   - If the producer produces items faster than the consumer can consume them, the buffer will fill up.
   
   - The producer will enter a waiting state ($\mathrm{sem\_wait}(\&empty)$), causing a discontinuity in its execution.

2. \textbf{Buffer Empty:}

   - If the consumer consumes items faster than the producer can produce them, the buffer will empty.
   
   - The consumer will enter a waiting state ($\mathrm{sem\_wait}(\&full)$), causing a discontinuity in its execution.

3. \textbf{Mutual Constraint Relationship:}

   - Both processes are constrained by the buffer’s state and must cooperate to achieve progress. This interdependency creates discontinuous execution as each process alternates between active and waiting states.

This example highlights the need for synchronization and how mutual constraints between concurrently executing programs lead to discontinuity in their execution.

\subsubsection{Bank Account Transactions with Concurrent Access}

\textbf{Scenario}

Consider a shared bank account with a balance that is accessed by two concurrent processes:

1. \textbf{Deposit Process}: Adds money to the account.

2. \textbf{Withdraw Process}: Deducts money from the account.

\textbf{Shared Resource}

- The \textbf{account balance} is a shared variable accessed by both processes.

\textbf{Discontinuity Characteristics}

1. \textbf{Insufficient Balance (Withdrawal Constraint):}

   - If the account balance is less than the withdrawal amount, the withdraw process cannot proceed. It must wait until the deposit process adds sufficient funds. This creates a discontinuity in the execution of the withdraw process.

2. \textbf{Concurrent Access (Race Condition):}

   - Without proper synchronization, both processes might try to read or modify the balance simultaneously. This can cause incorrect results (e.g., overdrawing the account). To prevent this, one process must wait while the other accesses the shared balance, introducing discontinuity in the waiting process.

\textbf{Detailed Example}

- Initial Balance: \$100

- Transactions:  

	- \textbf{Deposit Process} tries to deposit \$50.
	
  	- \textbf{Withdraw Process} tries to withdraw \$120.

\textbf{Execution Flow:}

1. Both processes start concurrently.

2. The withdraw process checks the balance (\$100) and finds it insufficient for the \$120 withdrawal.

3. The withdraw process pauses, waiting for the deposit process to increase the balance.

4. The deposit process completes its deposit, increasing the balance to \$150.

5. The withdraw process resumes and completes the \$120 withdrawal, leaving a balance of \$30.

\textbf{Mutual Constraint Relationship}

- The withdraw process is constrained by the balance being insufficient.

- The deposit process is constrained by ensuring consistency in updating the shared balance.

\textbf{Discontinuity in Execution}

- The withdraw process experiences discontinuity when it must wait for the balance to increase.

- The deposit process may also experience discontinuity if it has to wait for exclusive access to the shared balance during synchronization.

\textbf{Key Takeaways}

- The discontinuity arises from the \textbf{mutual constraint relationship} between the deposit and withdraw processes.

- Proper synchronization mechanisms, such as locks or semaphores, are necessary to manage access to the shared balance and avoid conflicts, which further introduces discontinuity when one process waits for the other.

\subsection{Closure and Reproducibility}

\begin{greenbox}
When a program is executed concurrently, it loses its closure and reproducibility. When a program is executed concurrently, multiple programs share various resources in the system, so the status of these resources is changed by multiple programs. This causes the program to lose its closure and also its reproducibility. What is closure? What is reproducibility? Please give an example to illustrate "When a program is executed concurrently, it loses its closure and reproducibility."
\end{greenbox}

当程序并发执行时,它就失去了封闭性和可重复性。当程序并发执行时,多个程序会共享系统中的各种资源,因此这些资源的状态会被多个程序改变。这会导致程序失去封闭性和可重复性。什么是封闭性?什么是可重复性?请举例说明“当程序并发执行时,它就失去了封闭性和可重复性。”

\subsubsection{Definitions}

1. \textbf{Closure}:  
   
   - \textbf{Definition}: Closure means that the behavior and outcome of a program are completely determined by its own code and input data, without being affected by any external factors.  
   
   - \textbf{Key Idea}: In a closed system, the program executes in isolation, and its execution is self-contained.

2. \textbf{Reproducibility}:  
   
   - \textbf{Definition}: Reproducibility means that given the same input and environment, a program will always produce the same output and behavior.  
   
   - \textbf{Key Idea}: The program’s results are consistent and predictable when re-executed under identical conditions.

\rule{0.75\textwidth}{0.5pt}

\subsubsection{Loss of Closure and Reproducibility in Concurrent Execution}

\textbf{Example: File Writing by Concurrent Programs}

\textbf{Scenario}:  
Two concurrent programs, \textbf{Program A} and \textbf{Program B}, are designed to write data to a shared log file.

- \textbf{Program A}: Writes the string "LogA" to the file.  

- \textbf{Program B}: Writes the string "LogB" to the same file.  

- Both programs run concurrently, sharing the same file resource.

\rule{0.75\textwidth}{0.5pt}

\textbf{Loss of Closure}:

- When executed concurrently, the programs no longer operate in isolation because they share the file resource.  

- Program A's behavior and output are influenced by Program B's access to the file, and vice versa.    - For example, if Program B writes to the file while Program A is writing, the data in the file might become interleaved or corrupted.
    
    - Result: Program A's output is no longer determined solely by its code and input.

\rule{0.75\textwidth}{0.5pt}

\textbf{Loss of Reproducibility}:

- The output of the shared log file depends on the scheduling and interleaving of operations between Program A and Program B.  

- If the programs are run multiple times under the same conditions:    - One execution might produce:

      LogA
      LogB
    
    - Another execution might produce:

      LogB
      LogA
   
    - Or even corrupted output like:

      LogALogB


- This happens because concurrent execution introduces non-deterministic factors such as thread scheduling, resource contention, or system interrupts.

\rule{0.75\textwidth}{0.5pt}

\textbf{Key Takeaways}

- \textbf{Loss of Closure}: Programs depend on shared resources (e.g., file, memory), so their behavior is influenced by other concurrent programs.

- \textbf{Loss of Reproducibility}: Outputs vary due to non-deterministic interleaving of concurrent processes.

\subsubsection{Solution to Preserve Closure and Reproducibility}

- Use synchronization mechanisms like \textbf{mutex locks} or \textbf{file access control} to ensure that only one program can access the shared resource at a time.

- However, this introduces additional complexity and potential performance issues, illustrating the inherent challenges of concurrent execution.

\subsection{Lifetime of a Process}

\begin{greenbox}
Dynamicity is the most basic characteristic of a process, which is generated by creation, executed by scheduling, suspended due to lack of resources, and destroyed by cancellation. A process has a certain life span, while a program is just a set of ordered instructions and is a static entity. Please use real-life examples to illustrate the creation, scheduling, lack of resources, and cancellation of a process from the perspective of the operating system.
\end{greenbox}

动态性是进程最基本的特性,表现为由创建产生,由调度执行,因得不到资源而暂停执行,由撤销而消亡。 进程有一定的生命期,而程序只是一组有序的指令集和,是静态实体。请从操作系统的视角,用现实生活中的例子体现进程的创建、调度、得不到资源和撤销。

\subsection{Real-Life Examples of Process Characteristics in the Operating System}

\textbf{1. Creation of a Process}

   - \textbf{Example}: Opening a web browser.
     
     - When a user clicks on the browser icon, the OS:
       
       - Loads the program code from storage into memory.
       
       - Initializes the process control block (PCB) with information about the process (e.g., memory allocation, CPU registers).
       
       - Marks the process as "ready" to execute.
     
     - This marks the creation of a new process.
     
\rule{0.75\textwidth}{0.5pt}


\textbf{2. Scheduling of a Process}

   - \textbf{Example}: Switching between a text editor and a music player.
     
     - The OS uses a CPU scheduler to decide which process gets CPU time:
       
       - If the text editor is currently active, the OS schedules it to run on the CPU.
       
       - Meanwhile, the music player runs in the background, scheduled periodically to ensure continuous playback.
     
     - Scheduling ensures fair distribution of CPU time among processes based on their priority and state.

\rule{0.75\textwidth}{0.5pt}

\textbf{3. Suspension Due to Lack of Resources}

   - \textbf{Example}: Waiting for a file to download.
     
     - A file download process may be paused (suspended) while waiting for network data.
       
       - The process enters a "waiting" state because the required resource (network data) is unavailable.
       
       - The OS removes it from the CPU queue and resumes another process that has resources available, like a text editor.
     
     - Once the network data arrives, the OS transitions the download process back to the "ready" state.

\rule{0.75\textwidth}{0.5pt}

\textbf{4. Cancellation of a Process}

\textbf{Example}: Force-closing an unresponsive application.

\begin{itemize}
\item 
If a program like a video player becomes unresponsive, the user might terminate it via the Task Manager (Windows) or Activity Monitor (macOS).
\begin{itemize}
\item The OS cancels the process by:

\begin{itemize}
\item
Releasing allocated resources (e.g., memory, file handles).
\item
Updating the process table to mark the process as "terminated."
\item
Cleaning up associated data to prevent resource leaks.
\end{itemize}
\item
This ends the life span of the process.

\end{itemize}
\end{itemize}

   - \textbf{Example}: Force-closing an unresponsive application.
     
     - If a program like a video player becomes unresponsive, the user might terminate it via the Task Manager (Windows) or Activity Monitor (macOS).
       
       - The OS cancels the process by:
         
         - Releasing allocated resources (e.g., memory, file handles).
         
         - Updating the process table to mark the process as "terminated."
         
         - Cleaning up associated data to prevent resource leaks.
     
     - This ends the life span of the process.

\rule{0.75\textwidth}{0.5pt}

\textbf{Summary of the Process Life Cycle in Real-Life Context}

- \textbf{Creation}: Starting a new application (e.g., launching a web browser).

- \textbf{Scheduling}: Alternating between tasks like editing a document and playing music.

- \textbf{Suspension}: Waiting for resources like network data or user input.

- \textbf{Cancellation}: Force-closing a stuck or unneeded program.
These examples illustrate how the OS dynamically manages processes to ensure efficient use of system resources while meeting user demands.

\end{document}
